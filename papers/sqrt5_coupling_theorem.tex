\documentclass[12pt,a4paper]{article}
\usepackage{amsmath,amssymb,amsthm}
\usepackage{geometry}
\usepackage{hyperref}
\usepackage{url}
\usepackage{booktabs}
\usepackage{bm}

\geometry{margin=1in}

\newtheorem{theorem}{Theorem}[section]
\newtheorem{corollary}[theorem]{Corollary}
\newtheorem{lemma}[theorem]{Lemma}
\newtheorem{proposition}[theorem]{Proposition}
\newtheorem{definition}[theorem]{Definition}
\newtheorem{remark}[theorem]{Remark}

\DeclareMathOperator{\diag}{diag}

\title{A $\sqrt{5}$-Coupling Identity for Golden-Ratio Folding Matrices\\
from E8 to H4}
\author{Paul Phillips\\
Clear Seas Solutions LLC\\
\texttt{Paul@clearseassolutions.com}}
\date{January 2026}

\begin{document}

\maketitle

\begin{abstract}
We study a class of $8 \times 8$ projection matrices that map the E8 root system to H4 polytope geometry via golden ratio coefficients. For matrices constructed with the coefficient triple $(a,b,c) = (\frac{1}{2}, \frac{\varphi-1}{2}, \frac{\varphi}{2})$, where $\varphi = (1+\sqrt{5})/2$ is the golden ratio, we prove that the row norms of the upper and lower $4 \times 8$ blocks satisfy a precise coupling identity: if $\|\text{H4L}\| = \sqrt{3-\varphi}$ and $\|\text{H4R}\| = \sqrt{\varphi+2}$ denote these row norms, then $\|\text{H4L}\| \cdot \|\text{H4R}\| = \sqrt{5}$. This follows from the elementary identity $(3-\varphi)(\varphi+2) = 5$. We further show that this $\sqrt{5}$-coupling persists as a $1/\sqrt{5}$ inner product between corresponding normalized rows. The result provides a precise algebraic characterization of the relationship between chiral components in E8$\to$H4 folding constructions.
\end{abstract}

\noindent\textbf{Keywords:} E8 lattice, H4 Coxeter group, golden ratio, folding projection, 600-cell, icosahedral symmetry

\noindent\textbf{MSC 2020:} 52B15 (Symmetry properties of polytopes), 20F55 (Reflection groups), 51M20 (Polyhedra and polytopes; regular figures)

\section{Introduction}

\subsection{Background}

The E8 root system is the unique even unimodular lattice in 8 dimensions and plays a distinguished role in both mathematics and mathematical physics \cite{conway1999}. Its 240 root vectors achieve the optimal sphere packing density in 8 dimensions \cite{viazovska2017}. The H4 Coxeter group, with its associated 600-cell and 120-cell polytopes in 4 dimensions, is the largest non-crystallographic reflection group and is characterized by the appearance of the golden ratio $\varphi = (1+\sqrt{5})/2$ in its geometric structure \cite{coxeter1973}.

A fundamental relationship exists between E8 and H4: the E8 root system can be ``folded'' or projected to produce H4 polytope vertices with golden ratio scaling. This connection has been studied by numerous authors \cite{conway1999, moxness2014, koca2001, baez2018}. Specifically, an $8 \times 8$ projection matrix can map the 240 roots of E8 to produce multiple copies of the 120-vertex 600-cell at scales related by powers of $\varphi$.

\subsection{Motivation}

While the E8$\to$H4 folding relationship is well-established, we observe that the explicit algebraic properties of the projection matrices themselves---particularly the relationships between their block structures---have not been fully characterized in the literature.

In this paper, we analyze a specific family of folding matrices constructed with golden-ratio-coupled coefficients. Our main result is the identification of a precise $\sqrt{5}$-coupling between the row norms of the ``left'' and ``right'' 4-dimensional subspaces. This coupling is not a numerical coincidence but follows directly from algebraic identities involving the golden ratio.

\subsection{Organization}

Section 2 reviews the necessary background on E8, H4, and folding projections. Section 3 defines the specific matrix family we study and derives its row norm properties. Section 4 states and proves our main theorem. Section 5 presents additional algebraic relationships. Section 6 discusses the geometric interpretation and places our result in context.

\section{Preliminaries}

\subsection{The E8 Root System}

\begin{definition}
The \textbf{E8 root system} consists of 240 vectors in $\mathbb{R}^8$ of two types:
\begin{enumerate}
    \item \textbf{Type I} (112 roots): All vectors of the form $(\pm 1, \pm 1, 0, 0, 0, 0, 0, 0)$ and permutations, where exactly two coordinates are $\pm 1$ and six are 0.
    \item \textbf{Type II} (128 roots): All vectors of the form $(\pm\frac{1}{2}, \pm\frac{1}{2}, \pm\frac{1}{2}, \pm\frac{1}{2}, \pm\frac{1}{2}, \pm\frac{1}{2}, \pm\frac{1}{2}, \pm\frac{1}{2})$ where the number of minus signs is even.
\end{enumerate}
All E8 roots have squared norm equal to 2.
\end{definition}

\begin{remark}
The E8 lattice is the unique even unimodular lattice in 8 dimensions up to isometry. Its root system has the remarkable property that its components lie in $\{0, \pm\frac{1}{2}, \pm 1\}$---notably, no irrational numbers appear.
\end{remark}

\subsection{The Golden Ratio and H4}

\begin{definition}
The \textbf{golden ratio} is $\varphi = \frac{1+\sqrt{5}}{2} \approx 1.618$. It satisfies the characteristic equation $\varphi^2 = \varphi + 1$.
\end{definition}

The following identities are used throughout this paper:

\begin{lemma}[Golden Ratio Identities]\label{lem:phi}
The golden ratio $\varphi$ satisfies:
\begin{enumerate}
    \item $\varphi^2 = \varphi + 1$
    \item $\varphi - \frac{1}{\varphi} = 1$
    \item $\varphi + \frac{1}{\varphi} = \sqrt{5}$
    \item $(\varphi - 1)^2 = \frac{1}{\varphi^2} = 2 - \varphi = \frac{3-\sqrt{5}}{2}$
    \item $\frac{1}{\varphi} = \varphi - 1 = \frac{\sqrt{5}-1}{2}$
\end{enumerate}
\end{lemma}

\begin{proof}
These follow directly from $\varphi = (1+\sqrt{5})/2$ and the characteristic equation. For (4): $(\varphi-1)^2 = \varphi^2 - 2\varphi + 1 = (\varphi+1) - 2\varphi + 1 = 2 - \varphi$.
\end{proof}

\begin{definition}
The \textbf{H4 Coxeter group} is the symmetry group of the 600-cell, a regular 4-dimensional polytope with 120 vertices. The 600-cell's vertex coordinates necessarily involve the golden ratio. The geometry of the 600-cell and its relationship to E8 has been studied extensively \cite{denney2019, baez2018}.
\end{definition}

\subsection{E8 to H4 Folding}

The E8$\to$H4 folding projection maps 8-dimensional E8 root vectors to 4-dimensional H4 geometry. This projection produces multiple copies of the 600-cell at different scales related by $\varphi$.

\begin{definition}
An \textbf{E8$\to$H4 folding matrix} is an $8 \times 8$ matrix $U$ such that when applied to the 240 E8 root vectors, the resulting 8-dimensional outputs can be partitioned into two 4-dimensional subspaces, each containing 600-cell vertices (possibly at multiple $\varphi$-related scales).
\end{definition}

Moxness \cite{moxness2014} constructed an explicit folding matrix using golden ratio coefficients. The present work analyzes the algebraic properties of such matrices.

\section{The Golden-Ratio Folding Matrix}

\subsection{Matrix Construction}

We study the following family of $8 \times 8$ projection matrices parameterized by golden-ratio coefficients.

\begin{definition}\label{def:coefficients}
Define the \textbf{golden-ratio coefficient triple}:
\begin{align}
    a &= \frac{1}{2} \\
    b &= \frac{1}{2\varphi} = \frac{\varphi - 1}{2} \approx 0.309 \\
    c &= \frac{\varphi}{2} \approx 0.809
\end{align}
These satisfy $b = a/\varphi$ and $c = a \cdot \varphi$.
\end{definition}

\begin{definition}\label{def:matrix}
The \textbf{$\varphi$-coupled folding matrix} $U$ is an $8 \times 8$ matrix partitioned into two $4 \times 8$ blocks:
\[
U = \begin{pmatrix} U_L \\ U_R \end{pmatrix}
\]
where $U_L$ (rows 0--3, the ``H4L block'') has entries from $\{0, \pm a, \pm b\}$ and $U_R$ (rows 4--7, the ``H4R block'') has entries from $\{0, \pm a, \pm c\}$.

The explicit matrix is:
\[
U = \begin{pmatrix}
a & b & a & b & a & -b & a & -b \\
a & a & -b & -b & -a & -a & b & b \\
a & -b & -a & b & a & -b & -a & b \\
a & -a & b & -b & -a & a & -b & b \\
c & a & c & a & c & -a & c & -a \\
c & c & -a & -a & -c & -c & a & a \\
c & -a & -c & a & c & -a & -c & a \\
c & -c & a & -a & -c & c & -a & a
\end{pmatrix}
\]
where $a = 1/2$, $b = (\varphi-1)/2$, $c = \varphi/2$.
\end{definition}

\begin{remark}
This matrix structure follows Moxness \cite{moxness2014}. Each row has exactly 8 non-zero entries: 4 entries of $\pm a$ and 4 entries of either $\pm b$ (for rows 0--3) or $\pm c$ (for rows 4--7).
\end{remark}

\subsection{Row Norm Computation}

\begin{proposition}\label{prop:norms}
For the $\varphi$-coupled folding matrix $U$:
\begin{enumerate}
    \item Each row of $U_L$ has squared norm $\|U_L\|^2 = 3 - \varphi = \frac{5-\sqrt{5}}{2} \approx 1.382$
    \item Each row of $U_R$ has squared norm $\|U_R\|^2 = \varphi + 2 = \frac{5+\sqrt{5}}{2} \approx 3.618$
\end{enumerate}
Thus $\|U_L\| = \sqrt{3-\varphi} \approx 1.176$ and $\|U_R\| = \sqrt{\varphi+2} \approx 1.902$.
\end{proposition}

\begin{proof}
\textbf{Part 1:} Each row of $U_L$ contains 4 entries of magnitude $|a| = 1/2$ and 4 entries of magnitude $|b| = (\varphi-1)/2$. Thus:
\begin{align}
    \|U_L\|^2 &= 4a^2 + 4b^2 \\
    &= 4 \cdot \frac{1}{4} + 4 \cdot \frac{(\varphi-1)^2}{4} \\
    &= 1 + (\varphi-1)^2
\end{align}
By Lemma \ref{lem:phi}(4), $(\varphi-1)^2 = 2 - \varphi$. Therefore:
\[
\|U_L\|^2 = 1 + (2-\varphi) = 3 - \varphi
\]
To verify algebraically: $3 - \varphi = 3 - \frac{1+\sqrt{5}}{2} = \frac{6-1-\sqrt{5}}{2} = \frac{5-\sqrt{5}}{2}$. \checkmark

\textbf{Part 2:} Each row of $U_R$ contains 4 entries of magnitude $|c| = \varphi/2$ and 4 entries of magnitude $|a| = 1/2$. Thus:
\begin{align}
    \|U_R\|^2 &= 4c^2 + 4a^2 \\
    &= 4 \cdot \frac{\varphi^2}{4} + 4 \cdot \frac{1}{4} \\
    &= \varphi^2 + 1
\end{align}
By Lemma \ref{lem:phi}(1), $\varphi^2 = \varphi + 1$. Therefore:
\[
\|U_R\|^2 = (\varphi+1) + 1 = \varphi + 2
\]
To verify algebraically: $\varphi + 2 = \frac{1+\sqrt{5}}{2} + 2 = \frac{5+\sqrt{5}}{2}$. \checkmark
\end{proof}

\section{The Main Result}

\begin{theorem}[$\sqrt{5}$-Coupling Theorem]\label{thm:main}
For the $\varphi$-coupled folding matrix $U$ with coefficient triple $(a,b,c) = (\frac{1}{2}, \frac{\varphi-1}{2}, \frac{\varphi}{2})$, the row norms of the H4L and H4R blocks satisfy:
\[
\|U_L\| \cdot \|U_R\| = \sqrt{5}
\]
Equivalently, $(3-\varphi)(\varphi+2) = 5$.
\end{theorem}

\begin{proof}
By Proposition \ref{prop:norms}, $\|U_L\|^2 = 3-\varphi$ and $\|U_R\|^2 = \varphi+2$. We must show:
\[
(3-\varphi)(\varphi+2) = 5
\]

\textbf{Direct computation:}
\begin{align}
    (3-\varphi)(\varphi+2) &= 3\varphi + 6 - \varphi^2 - 2\varphi \\
    &= \varphi + 6 - \varphi^2
\end{align}

Substituting $\varphi^2 = \varphi + 1$ (Lemma \ref{lem:phi}(1)):
\begin{align}
    &= \varphi + 6 - (\varphi + 1) \\
    &= \varphi + 6 - \varphi - 1 \\
    &= 5
\end{align}

Therefore $\|U_L\| \cdot \|U_R\| = \sqrt{(3-\varphi)(\varphi+2)} = \sqrt{5}$.
\end{proof}

\begin{remark}
The identity $(3-\varphi)(\varphi+2) = 5$ can also be verified using the representation $\varphi = (1+\sqrt{5})/2$:
\begin{align}
    (3-\varphi)(\varphi+2) &= \frac{5-\sqrt{5}}{2} \cdot \frac{5+\sqrt{5}}{2} \\
    &= \frac{(5-\sqrt{5})(5+\sqrt{5})}{4} \\
    &= \frac{25 - 5}{4} = \frac{20}{4} = 5 \checkmark
\end{align}
This is a difference-of-squares factorization.
\end{remark}

\subsection{Inter-Block Inner Products}

The $\sqrt{5}$-coupling also appears in the inner products between rows of $U_L$ and $U_R$.

\begin{proposition}\label{prop:innerproduct}
For the $\varphi$-coupled folding matrix, the inner product between corresponding rows of $U_L$ and $U_R$ (e.g., Row 0 and Row 4) equals $\sqrt{5}$.
\end{proposition}

\begin{proof}
From Definition \ref{def:matrix}, Row 0 $= (a, b, a, b, a, -b, a, -b)$ and Row 4 $= (c, a, c, a, c, -a, c, -a)$.

Computing the dot product term by term:
\begin{align}
    \text{Row}_0 \cdot \text{Row}_4 &= (a)(c) + (b)(a) + (a)(c) + (b)(a) + (a)(c) + (-b)(-a) + (a)(c) + (-b)(-a) \\
    &= 4ac + 4ab \\
    &= 4a(b+c)
\end{align}

Substituting the coefficient values:
\[
b + c = \frac{\varphi-1}{2} + \frac{\varphi}{2} = \frac{2\varphi - 1}{2}
\]

Therefore:
\[
4a(b+c) = 4 \cdot \frac{1}{2} \cdot \frac{2\varphi-1}{2} = 2\varphi - 1
\]

Since $\varphi = (1+\sqrt{5})/2$:
\[
2\varphi - 1 = 2 \cdot \frac{1+\sqrt{5}}{2} - 1 = 1 + \sqrt{5} - 1 = \sqrt{5}
\]
\end{proof}

\begin{corollary}[$\varphi$-Scaling]\label{cor:phiscaling}
Corresponding rows of $U_L$ and $U_R$ are related by the golden ratio:
\[
\text{Row}_L = \frac{1}{\varphi} \cdot \text{Row}_R
\]
Consequently:
\begin{enumerate}
    \item $\|U_L\| = \frac{1}{\varphi} \|U_R\|$
    \item $\text{Row}_L \cdot \text{Row}_R = \frac{1}{\varphi} \|U_R\|^2 = \frac{\varphi+2}{\varphi} = \sqrt{5}$
    \item After normalization, $\hat{U}_L = \hat{U}_R$ (the rows become identical)
\end{enumerate}
\end{corollary}

\begin{proof}
The coefficients satisfy $a/c = 1/\varphi$ and $b/a = \varphi - 1 = 1/\varphi$, so every entry in Row~$i$ of $U_L$ equals $1/\varphi$ times the corresponding entry in Row~$(i+4)$ of $U_R$. Part (1) follows immediately. For part (2):
\[
\text{Row}_L \cdot \text{Row}_R = \frac{1}{\varphi} \|U_R\|^2 = \frac{\varphi+2}{\varphi} = 1 + \frac{2}{\varphi} = 1 + 2(\varphi-1) = 2\varphi - 1 = \sqrt{5}
\]
Part (3) follows since scaling by a positive constant does not change direction.
\end{proof}

\begin{remark}
This $\varphi$-scaling relationship provides an alternative proof of Theorem \ref{thm:main}: since $\|U_L\| = \|U_R\|/\varphi$, we have $\|U_L\| \cdot \|U_R\| = \|U_R\|^2/\varphi = (\varphi+2)/\varphi = \sqrt{5}$.
\end{remark}

\section{The $\varphi$-Hierarchy of Norms}

The row norms $\sqrt{3-\varphi}$ and $\sqrt{\varphi+2}$ fit into a larger algebraic hierarchy of values that appear naturally in E8$\to$H4 geometry.

\begin{proposition}
The following table shows algebraically significant values arising in E8$\to$H4 projections, all expressible in terms of $\varphi$ and $\sqrt{5}$:
\end{proposition}

\begin{center}
\begin{tabular}{cccl}
\toprule
Value & Decimal & Algebraic Form & Description \\
\midrule
$\varphi^{-2}$ & 0.382 & $2 - \varphi = \frac{3-\sqrt{5}}{2}$ & Scaling factor \\[2pt]
$\varphi^{-1}$ & 0.618 & $\varphi - 1 = \frac{\sqrt{5}-1}{2}$ & Golden ratio conjugate \\[2pt]
$1$ & 1.000 & $1$ & Unit \\[2pt]
$\sqrt{3-\varphi}$ & 1.176 & $\sqrt{\frac{5-\sqrt{5}}{2}}$ & \textbf{H4L row norm} \\[2pt]
$\sqrt{2}$ & 1.414 & $\sqrt{2}$ & E8 root norm \\[2pt]
$\varphi$ & 1.618 & $\frac{1+\sqrt{5}}{2}$ & Golden ratio \\[2pt]
$\sqrt{\varphi+2}$ & 1.902 & $\sqrt{\frac{5+\sqrt{5}}{2}}$ & \textbf{H4R row norm} \\[2pt]
$\varphi^2$ & 2.618 & $\varphi + 1 = \frac{3+\sqrt{5}}{2}$ & Golden ratio squared \\[2pt]
\bottomrule
\end{tabular}
\end{center}

\begin{remark}
The row norms $\sqrt{3-\varphi}$ and $\sqrt{\varphi+2}$ can be rewritten as:
\begin{align}
    \sqrt{3-\varphi} &= \sqrt{\frac{5-\sqrt{5}}{2}} = \frac{\sqrt{10-2\sqrt{5}}}{2} \\
    \sqrt{\varphi+2} &= \sqrt{\frac{5+\sqrt{5}}{2}} = \frac{\sqrt{10+2\sqrt{5}}}{2}
\end{align}
These are the radii of certain circles inscribed in regular pentagons, connecting to well-known constructions in icosahedral geometry.
\end{remark}

The following identity unifies the hierarchy:

\begin{proposition}
The product-sum identity:
\[
\sqrt{3-\varphi} \cdot \sqrt{\varphi+2} = \sqrt{5} = \varphi + \varphi^{-1}
\]
\end{proposition}

\begin{proof}
The left equality is Theorem \ref{thm:main}. For the right equality:
\[
\varphi + \varphi^{-1} = \frac{1+\sqrt{5}}{2} + \frac{2}{1+\sqrt{5}} = \frac{1+\sqrt{5}}{2} + \frac{\sqrt{5}-1}{2} = \frac{2\sqrt{5}}{2} = \sqrt{5}
\]
This is Lemma \ref{lem:phi}(3).
\end{proof}

\section{Computational Verification}

The algebraic results in this paper have been independently verified by numerical computation to machine precision ($\approx 10^{-15}$).

\begin{center}
\begin{tabular}{lcc}
\toprule
Identity & Algebraic & Numerical \\
\midrule
$\varphi^2 - \varphi - 1$ & $= 0$ & $2.2 \times 10^{-16}$ \\
$\varphi - 1/\varphi - 1$ & $= 0$ & $0$ \\
$(3-\varphi)(\varphi+2) - 5$ & $= 0$ & $0$ \\
$\|U_L\|^2 - (3-\varphi)$ & $= 0$ & $0$ \\
$\|U_R\|^2 - (\varphi+2)$ & $= 0$ & $0$ \\
$\|U_L\| \cdot \|U_R\| - \sqrt{5}$ & $= 0$ & $0$ \\
\bottomrule
\end{tabular}
\end{center}

The computation was performed in double-precision floating point arithmetic using the exact matrix entries from Definition \ref{def:matrix}.

\section{Discussion}

\subsection{The Fundamental Structure}

Our analysis reveals that the $\sqrt{5}$-coupling (Theorem \ref{thm:main}) is a consequence of a more fundamental fact: the H4L and H4R row blocks are related by a $\varphi$-scaling (Corollary \ref{cor:phiscaling}). Specifically:
\[
U_L = \frac{1}{\varphi} U_R
\]
where corresponding rows are scaled copies. This explains why the folding matrix produces two H4 copies at different scales---they are geometrically identical but scaled by the golden ratio.

\subsection{Relationship to Moxness's Work}

Moxness \cite{moxness2014} introduced folding matrices for E8$\to$H4 projection in the context of visualization. The coefficient triple $(a,b,c) = (1/2, (\varphi-1)/2, \varphi/2)$ was chosen to produce the golden-ratio scaling between H4 copies. Our contribution is the precise algebraic characterization of this scaling: the coefficients form a geometric sequence with common ratio $\varphi$, which directly implies the $\sqrt{5}$-coupling.

To our knowledge, the specific identities $\|U_L\| \cdot \|U_R\| = \sqrt{5}$ and $\text{Row}_L = (1/\varphi)\text{Row}_R$ have not been previously published.

\subsection{Geometric Interpretation}

The appearance of $\sqrt{5}$ as the coupling constant reflects its fundamental role in golden ratio geometry: $\sqrt{5} = \varphi + \varphi^{-1}$. The two 4-dimensional subspaces (H4L and H4R) are not independent but are $\varphi$-scaled copies embedded in the 8-dimensional E8 space. After orthonormalization, they become identical, indicating that the ``two'' H4 copies are really one copy viewed at two scales.

This structure is consistent with the known decomposition of E8 roots into four H4 copies at scales $1, \varphi, \varphi^{-1}, 1$ \cite{conway1999}.

\subsection{Open Questions}

Several questions remain:
\begin{enumerate}
    \item Does every E8$\to$H4 folding matrix (regardless of coefficient choice) exhibit a similar coupling identity?
    \item What is the analogous structure for other exceptional folding relationships (e.g., E6$\to$G2)?
    \item How does this $\varphi$-scaling manifest in the projected 600-cell vertex structure?
\end{enumerate}

\section{Conclusion}

We have analyzed a family of golden-ratio folding matrices that project the E8 root system to H4 polytope geometry. For the coefficient triple $(a,b,c) = (1/2, (\varphi-1)/2, \varphi/2)$, we proved two related results:

\textbf{Main Theorem (Theorem \ref{thm:main}):} The row norms satisfy $\|U_L\| \cdot \|U_R\| = \sqrt{5}$, where the individual norms are $\|U_L\| = \sqrt{3-\varphi}$ and $\|U_R\| = \sqrt{\varphi+2}$.

\textbf{Structural Insight (Corollary \ref{cor:phiscaling}):} The H4L and H4R blocks are $\varphi$-scaled copies of each other: $U_L = (1/\varphi) U_R$. This implies the $\sqrt{5}$-coupling and explains the geometric relationship between the two 4-dimensional subspaces.

These results follow from elementary algebraic identities involving the golden ratio, particularly $(3-\varphi)(\varphi+2) = 5$ and $\varphi - 1 = 1/\varphi$.

The $\varphi$-scaling structure suggests that E8$\to$H4 folding does not produce two independent H4 copies, but rather one H4 structure viewed at two $\varphi$-related scales---a manifestation of the self-similar properties inherent in golden ratio geometry.

To our knowledge, the specific identities $\|U_L\| \cdot \|U_R\| = \sqrt{5}$ and $U_L = (1/\varphi) U_R$ have not been previously published.

\section*{Acknowledgments}

The author thanks J.G. Moxness for foundational work on E8$\to$H4 folding matrices.

\begin{thebibliography}{99}

\bibitem{conway1999}
J.H. Conway and N.J.A. Sloane,
\textit{Sphere Packings, Lattices and Groups},
Grundlehren der mathematischen Wissenschaften, vol.~290, 3rd ed.
Springer, New York, 1999.
ISBN: 978-0-387-98585-5.
DOI: 10.1007/978-1-4757-6568-7.

\bibitem{coxeter1973}
H.S.M. Coxeter,
\textit{Regular Polytopes}, 3rd ed.
Dover Publications, New York, 1973.
ISBN: 978-0-486-61480-9.

\bibitem{viazovska2017}
M. Viazovska,
``The sphere packing problem in dimension 8,''
\textit{Annals of Mathematics}, vol.~185, no.~3, pp.~991--1015, 2017.
DOI: 10.4007/annals.2017.185.3.7.

\bibitem{moxness2014}
J.G. Moxness,
``The 3D Visualization of E8 using an H4 Folding Matrix,''
viXra:1411.0130, 2014.
Available: \url{https://vixra.org/abs/1411.0130}.

\bibitem{koca2001}
M. Koca, R. Ko\c{c}, and M. Al-Barwani,
``Noncrystallographic Coxeter group H4 in E8,''
\textit{Journal of Physics A: Mathematical and General}, vol.~34, no.~50, pp.~11201--11213, 2001.
DOI: 10.1088/0305-4470/34/50/303.

\bibitem{denney2019}
T. Denney, D. Hooker, D. Johnson, T. Robinson, M. Butler, and S. Claiborne,
``The Geometry of H4 Polytopes,''
arXiv:1912.06156 [math.MG], 2019.
DOI: 10.48550/arXiv.1912.06156.

\bibitem{baez2018}
J.C. Baez,
``From the Icosahedron to E8,''
\textit{London Mathematical Society Newsletter}, no.~476, pp.~18--23, 2018.
arXiv:1712.06436.

\end{thebibliography}

\end{document}
