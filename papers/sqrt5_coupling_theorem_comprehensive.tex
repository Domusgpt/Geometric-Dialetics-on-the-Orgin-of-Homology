\documentclass[12pt,a4paper]{article}

% ============================================================
% PACKAGES
% ============================================================
\usepackage{amsmath,amssymb,amsthm,mathtools}
\usepackage{geometry}
\usepackage{hyperref}
\usepackage{url}
\usepackage{booktabs}
\usepackage{bm}
\usepackage{tikz}
\usepackage{tikz-cd}
\usepackage{pgfplots}
\usepackage{float}
\usepackage{enumitem}
\usepackage{xcolor}
\usepackage{tcolorbox}
\usepackage{listings}
\usepackage{appendix}

\pgfplotsset{compat=1.17}
\usetikzlibrary{matrix,positioning,arrows.meta,calc}

\geometry{margin=1in}

% ============================================================
% THEOREM ENVIRONMENTS
% ============================================================
\newtheorem{theorem}{Theorem}[section]
\newtheorem{corollary}[theorem]{Corollary}
\newtheorem{lemma}[theorem]{Lemma}
\newtheorem{proposition}[theorem]{Proposition}
\newtheorem{conjecture}[theorem]{Conjecture}

\theoremstyle{definition}
\newtheorem{definition}[theorem]{Definition}
\newtheorem{example}[theorem]{Example}
\newtheorem{construction}[theorem]{Construction}

\theoremstyle{remark}
\newtheorem{remark}[theorem]{Remark}
\newtheorem{notation}[theorem]{Notation}
\newtheorem{observation}[theorem]{Observation}

% ============================================================
% CUSTOM COMMANDS
% ============================================================
\newcommand{\R}{\mathbb{R}}
\newcommand{\Z}{\mathbb{Z}}
\newcommand{\Q}{\mathbb{Q}}
\newcommand{\C}{\mathbb{C}}
\newcommand{\HH}{\mathbb{H}}
\newcommand{\norm}[1]{\left\|#1\right\|}
\newcommand{\abs}[1]{\left|#1\right|}
\newcommand{\inner}[2]{\langle #1, #2 \rangle}
\newcommand{\ph}{\varphi}
\newcommand{\phib}{\bar{\varphi}}
\DeclareMathOperator{\diag}{diag}
\DeclareMathOperator{\Tr}{Tr}
\DeclareMathOperator{\sgn}{sgn}

% Code listing style
\lstset{
  language=Python,
  basicstyle=\small\ttfamily,
  keywordstyle=\color{blue},
  commentstyle=\color{gray},
  numbers=left,
  numberstyle=\tiny\color{gray},
  frame=single,
  breaklines=true
}

% ============================================================
% TITLE
% ============================================================
\title{The $\sqrt{5}$-Coupling Theorem:\\
Golden Ratio Structure in E8$\to$H4 Folding Projections\\[10pt]
\large A Comprehensive Mathematical Analysis}

\author{Paul Phillips\\
Clear Seas Solutions LLC\\
\texttt{Paul@clearseassolutions.com}\\[10pt]
\small Parserator.com}

\date{January 2026}

\begin{document}

\maketitle

% ============================================================
% ABSTRACT
% ============================================================
\begin{abstract}
We present a comprehensive analysis of golden-ratio folding matrices that project the 240-root E8 lattice to four-dimensional H4 polytope geometry. For the specific coefficient family $(a,b,c) = (\frac{1}{2}, \frac{\varphi-1}{2}, \frac{\varphi}{2})$ where $\varphi = (1+\sqrt{5})/2$ is the golden ratio, we prove that the $8 \times 8$ projection matrix $U$ decomposes into upper (H4L) and lower (H4R) blocks whose row norms satisfy a precise coupling identity:
\[
\norm{U_L} \cdot \norm{U_R} = \sqrt{5}
\]
where $\norm{U_L} = \sqrt{3-\varphi} \approx 1.176$ and $\norm{U_R} = \sqrt{\varphi+2} \approx 1.902$. This follows from the elementary identity $(3-\varphi)(\varphi+2) = 5$.

More fundamentally, we prove that the blocks satisfy a $\varphi$-scaling relationship: $U_L = \frac{1}{\varphi} U_R$, meaning corresponding rows are related by the golden ratio. This implies that after orthonormalization, the ``two'' H4 copies produced by the folding are geometrically identical---a manifestation of the self-similar structure inherent in icosahedral geometry.

We provide four independent proofs of the main theorem, extensive computational verification, worked examples, and place the result in the context of the broader E8-H4-icosahedral connection that links the golden ratio, quaternions, the 600-cell, and exceptional mathematical structures.
\end{abstract}

\vspace{0.5cm}
\noindent\textbf{Keywords:} E8 lattice, H4 Coxeter group, golden ratio, folding projection, 600-cell, icosahedral symmetry, quaternions, non-crystallographic symmetry

\vspace{0.3cm}
\noindent\textbf{MSC 2020:} 52B15 (Symmetry properties of polytopes), 20F55 (Reflection and Coxeter groups), 51M20 (Polyhedra and polytopes), 17B22 (Root systems), 11H06 (Lattices and convex bodies)

\tableofcontents

% ============================================================
% SECTION 1: INTRODUCTION
% ============================================================
\section{Introduction}

\subsection{Overview and Main Results}

The exceptional structures of mathematics---the E8 lattice, icosahedral symmetry, and the golden ratio---are deeply interconnected. The E8 root system, the densest sphere packing in 8 dimensions, can be ``folded'' or projected to produce the vertices of four-dimensional polytopes with icosahedral (H4) symmetry. This folding introduces the golden ratio $\varphi = (1+\sqrt{5})/2$ into structures that originally contained only rational coordinates.

This paper analyzes the algebraic structure of the folding matrices themselves. Our main results are:

\begin{tcolorbox}[colback=blue!5,colframe=blue!50!black,title=Main Results]
\begin{theorem}[$\sqrt{5}$-Coupling Theorem]\label{thm:main-intro}
For the $\varphi$-coupled E8$\to$H4 folding matrix with coefficient triple $(a,b,c) = (\frac{1}{2}, \frac{\varphi-1}{2}, \frac{\varphi}{2})$:
\[
\norm{U_L} \cdot \norm{U_R} = \sqrt{5}
\]
where $\norm{U_L} = \sqrt{3-\varphi}$ and $\norm{U_R} = \sqrt{\varphi+2}$ are the row norms of the upper and lower $4 \times 8$ blocks.
\end{theorem}

\begin{theorem}[$\varphi$-Scaling Theorem]\label{thm:scaling-intro}
The blocks satisfy:
\[
U_L = \frac{1}{\varphi} \cdot U_R
\]
That is, every entry in row $i$ of $U_L$ equals $1/\varphi$ times the corresponding entry in row $(i+4)$ of $U_R$.
\end{theorem}
\end{tcolorbox}

\subsection{Historical Context}

The connection between the icosahedron and E8 has been recognized since at least the work of Conway and Sloane \cite{conway1999}, who described the ``icosians''---a subring of the quaternions related to icosahedral symmetry---and showed that the icosian lattice is isometric to E8. This connection was further developed by:

\begin{itemize}
\item \textbf{Koca et al.\ (2001)} \cite{koca2001}: Explicit embedding of the non-crystallographic Coxeter group H4 into W(E8) using quaternionic representations.
\item \textbf{Moxness (2014)} \cite{moxness2014}: Construction of explicit $8 \times 8$ folding matrices for E8$\to$H4 projection, enabling visualization of E8 structure.
\item \textbf{Baez (2018)} \cite{baez2018}: Two constructions of E8 from the icosahedron---via icosians and via resolution of Kleinian singularities.
\item \textbf{Denney et al.\ (2019)} \cite{denney2019}: Detailed geometry of H4 polytopes and the arrangement of 24-cells within the 600-cell.
\end{itemize}

While the \emph{existence} of E8$\to$H4 folding is well-established, the \emph{algebraic properties of specific folding matrices} have not been fully characterized. This paper fills that gap for the $\varphi$-coupled family.

\subsection{Why $\sqrt{5}$?}

The appearance of $\sqrt{5}$ is not accidental. The golden ratio satisfies:
\[
\varphi = \frac{1+\sqrt{5}}{2}, \qquad \varphi + \varphi^{-1} = \sqrt{5}, \qquad \varphi - \varphi^{-1} = 1
\]

Thus $\sqrt{5}$ is the fundamental irrational underlying $\varphi$. Our theorem shows that this same $\sqrt{5}$ governs the coupling between chiral H4 components in the folding projection---a precise algebraic manifestation of icosahedral geometry's self-similar structure.

\subsection{Organization}

The paper is organized as follows:

\begin{itemize}
\item \textbf{Section 2}: Mathematical preliminaries on the golden ratio, quaternions, E8, and H4.
\item \textbf{Section 3}: Construction of the $\varphi$-coupled folding matrix with detailed derivations.
\item \textbf{Section 4}: Four independent proofs of the $\sqrt{5}$-Coupling Theorem.
\item \textbf{Section 5}: The $\varphi$-Scaling Theorem and its consequences.
\item \textbf{Section 6}: Worked examples with explicit calculations.
\item \textbf{Section 7}: Computational verification and numerical validation.
\item \textbf{Section 8}: Discussion, connections to broader theory, and open questions.
\item \textbf{Appendix A}: Complete proof details.
\item \textbf{Appendix B}: Python verification code.
\item \textbf{Appendix C}: Tables of computed values.
\end{itemize}

% ============================================================
% SECTION 2: MATHEMATICAL PRELIMINARIES
% ============================================================
\section{Mathematical Preliminaries}

This section provides complete background on all mathematical structures used in this paper. Readers familiar with these topics may skip to Section 3.

\subsection{The Golden Ratio}

\begin{definition}[Golden Ratio]
The \textbf{golden ratio} is the positive root of $x^2 - x - 1 = 0$:
\[
\varphi = \frac{1 + \sqrt{5}}{2} \approx 1.6180339887
\]
Its algebraic conjugate is:
\[
\phib = \frac{1 - \sqrt{5}}{2} = -\frac{1}{\varphi} = 1 - \varphi \approx -0.6180339887
\]
\end{definition}

\begin{lemma}[Fundamental Golden Ratio Identities]\label{lem:phi-identities}
The golden ratio satisfies:
\begin{enumerate}[label=(\alph*)]
\item $\varphi^2 = \varphi + 1$ \hfill (defining property)
\item $1/\varphi = \varphi - 1$ \hfill (reciprocal)
\item $\varphi + 1/\varphi = \sqrt{5}$ \hfill (sum)
\item $\varphi - 1/\varphi = 1$ \hfill (difference)
\item $\varphi \cdot \phib = -1$ \hfill (product of conjugates)
\item $\varphi + \phib = 1$ \hfill (sum of conjugates)
\item $(\varphi - 1)^2 = 1/\varphi^2 = 2 - \varphi$ \hfill (squared inverse)
\item $\varphi^n = F_n \varphi + F_{n-1}$ for Fibonacci numbers $F_n$ \hfill (Fibonacci connection)
\end{enumerate}
\end{lemma}

\begin{proof}
(a) By definition. (b) From $\varphi^2 = \varphi + 1$, divide by $\varphi$: $\varphi = 1 + 1/\varphi$, so $1/\varphi = \varphi - 1$. (c) $\varphi + 1/\varphi = \varphi + (\varphi - 1) = 2\varphi - 1 = 2 \cdot \frac{1+\sqrt{5}}{2} - 1 = \sqrt{5}$. (d) $\varphi - 1/\varphi = \varphi - (\varphi - 1) = 1$. (e)-(h) Follow from the characteristic equation. \qed
\end{proof}

\begin{notation}
Throughout this paper, we write $\varphi$ for the golden ratio and reserve $\phi$ for angles. We use the identities in Lemma \ref{lem:phi-identities} freely without further reference.
\end{notation}

\begin{example}[Powers of $\varphi$]\label{ex:phi-powers}
The first several powers:
\begin{center}
\begin{tabular}{cccl}
\toprule
$n$ & $\varphi^n$ & Decimal & Fibonacci Form \\
\midrule
$-2$ & $2 - \varphi$ & $0.382$ & $F_{-2}\varphi + F_{-3} = -\varphi + 2$ \\
$-1$ & $\varphi - 1$ & $0.618$ & $F_{-1}\varphi + F_{-2} = \varphi - 1$ \\
$0$ & $1$ & $1.000$ & $F_0\varphi + F_{-1} = 1$ \\
$1$ & $\varphi$ & $1.618$ & $F_1\varphi + F_0 = \varphi$ \\
$2$ & $\varphi + 1$ & $2.618$ & $F_2\varphi + F_1 = \varphi + 1$ \\
$3$ & $2\varphi + 1$ & $4.236$ & $F_3\varphi + F_2 = 2\varphi + 1$ \\
\bottomrule
\end{tabular}
\end{center}
\end{example}

\subsection{The Ring $\Z[\varphi]$}

\begin{definition}
The \textbf{ring of golden integers} is:
\[
\Z[\varphi] = \{a + b\varphi : a, b \in \Z\}
\]
This is the ring of integers of the quadratic field $\Q(\sqrt{5})$.
\end{definition}

\begin{proposition}
$\Z[\varphi]$ is closed under addition, subtraction, and multiplication. The \textbf{norm} function $N: \Z[\varphi] \to \Z$ defined by $N(a + b\varphi) = (a + b\varphi)(a + b\phib) = a^2 + ab - b^2$ is multiplicative.
\end{proposition}

\subsection{Quaternions}

\begin{definition}[Quaternions]
The \textbf{quaternion algebra} $\HH$ consists of elements $q = a + bi + cj + dk$ where $a,b,c,d \in \R$ and:
\[
i^2 = j^2 = k^2 = ijk = -1
\]
The conjugate is $\bar{q} = a - bi - cj - dk$ and the norm is $\norm{q}^2 = q\bar{q} = a^2 + b^2 + c^2 + d^2$.
\end{definition}

\begin{definition}[Unit Quaternions]
The \textbf{unit quaternions} $\{q \in \HH : \norm{q} = 1\}$ form the 3-sphere $S^3$ and are isomorphic to the Lie group $\mathrm{SU}(2)$.
\end{definition}

\subsection{The Binary Icosahedral Group}

\begin{definition}
The \textbf{binary icosahedral group} $2I$ (also denoted $\tilde{A}_5$ or $\Gamma$) is the preimage of the icosahedral rotation group $A_5 \subset \mathrm{SO}(3)$ under the 2:1 covering $\mathrm{SU}(2) \to \mathrm{SO}(3)$. It has 120 elements.
\end{definition}

\begin{proposition}[Explicit Elements of $2I$]
The 120 elements of $2I$ can be written as unit quaternions:
\begin{itemize}
\item 8 elements: $\pm 1, \pm i, \pm j, \pm k$
\item 16 elements: $\frac{1}{2}(\pm 1 \pm i \pm j \pm k)$ (even number of minus signs)
\item 96 elements: Even permutations of $\frac{1}{2}(0, \pm 1, \pm \varphi^{-1}, \pm \varphi)$
\end{itemize}
\end{proposition}

\begin{definition}[Icosians]
The \textbf{icosians} are integer linear combinations of the 120 elements of $2I$ over $\Z$:
\[
\mathbb{I} = \Z[2I] = \left\{\sum_{g \in 2I} n_g \cdot g : n_g \in \Z\right\}
\]
\end{definition}

\begin{theorem}[Conway-Sloane \cite{conway1999}]
The icosians $\mathbb{I}$, viewed as 8-tuples via the identification $a + bi + cj + dk \mapsto (a_1, a_2, b_1, b_2, c_1, c_2, d_1, d_2)$ where each coefficient is written $x = x_1 + x_2\varphi$ with $x_1, x_2 \in \Z$, form a lattice isometric to the E8 root lattice.
\end{theorem}

\subsection{The E8 Root System}

\begin{definition}[E8 Root Lattice]
The \textbf{E8 root lattice} is the set of vectors $\bm{x} = (x_1, \ldots, x_8) \in \R^8$ satisfying:
\begin{enumerate}
\item All $x_i \in \Z$ or all $x_i \in \Z + \frac{1}{2}$
\item $\sum_{i=1}^8 x_i \equiv 0 \pmod{2}$
\end{enumerate}
\end{definition}

\begin{proposition}[E8 Root System]
The E8 lattice has 240 \textbf{roots} (minimal nonzero vectors), each with squared norm 2:
\begin{itemize}
\item \textbf{Type I} (112 roots): All permutations of $(\pm 1, \pm 1, 0, 0, 0, 0, 0, 0)$
\item \textbf{Type II} (128 roots): All vectors $(\pm\frac{1}{2}, \pm\frac{1}{2}, \pm\frac{1}{2}, \pm\frac{1}{2}, \pm\frac{1}{2}, \pm\frac{1}{2}, \pm\frac{1}{2}, \pm\frac{1}{2})$ with an even number of minus signs
\end{itemize}
\end{proposition}

\begin{theorem}[Viazovska 2016 \cite{viazovska2017}]
The E8 lattice achieves the optimal sphere packing density in 8 dimensions.
\end{theorem}

\begin{observation}\label{obs:e8-no-phi}
The E8 root coordinates lie in $\{0, \pm\frac{1}{2}, \pm 1\}$. No irrational numbers appear. The golden ratio emerges only upon projection to H4.
\end{observation}

\subsection{The H4 Coxeter Group and 600-Cell}

\begin{definition}[H4 Coxeter Group]
The \textbf{H4 Coxeter group} is the symmetry group of the 600-cell, the regular 4-dimensional polytope with:
\begin{itemize}
\item 120 vertices
\item 720 edges
\item 1200 triangular faces
\item 600 tetrahedral cells
\end{itemize}
It is the largest non-crystallographic (involving $\varphi$) reflection group.
\end{definition}

\begin{proposition}[600-Cell Vertices]
The 120 vertices of the 600-cell (with unit edge length) can be given as:
\begin{itemize}
\item 8 vertices: Permutations of $(\pm 1, 0, 0, 0)$
\item 16 vertices: $\frac{1}{2}(\pm 1, \pm 1, \pm 1, \pm 1)$
\item 96 vertices: Even permutations of $\frac{1}{2}(\pm\varphi, \pm 1, \pm\varphi^{-1}, 0)$
\end{itemize}
These are precisely the unit quaternions in the binary icosahedral group.
\end{proposition}

\begin{theorem}[E8 $\to$ H4 Decomposition \cite{koca2001}]
The 240 E8 roots, when projected to 4 dimensions via an appropriate folding matrix, decompose into:
\[
\text{E8 roots} \xrightarrow{\text{fold}} \text{H4}_L \oplus \varphi\text{H4}_L \oplus \text{H4}_R \oplus \varphi\text{H4}_R
\]
Four copies of the 600-cell vertices at two scales ($1$ and $\varphi$), partitioned into left (L) and right (R) chiralities.
\end{theorem}

% ============================================================
% SECTION 3: THE FOLDING MATRIX
% ============================================================
\section{Construction of the $\varphi$-Coupled Folding Matrix}

\subsection{The Coefficient Family}

\begin{definition}[Golden-Ratio Coefficient Triple]\label{def:coefficients}
Define the \textbf{$\varphi$-coupled coefficients}:
\begin{align}
a &= \frac{1}{2} \\
b &= \frac{1}{2\varphi} = \frac{\varphi - 1}{2} \approx 0.30902 \\
c &= \frac{\varphi}{2} \approx 0.80902
\end{align}
\end{definition}

\begin{observation}[Geometric Sequence]
The coefficients form a geometric sequence with common ratio $\varphi$:
\[
b : a : c = 1 : \varphi : \varphi^2
\]
Explicitly: $b \cdot \varphi = a$ and $a \cdot \varphi = c$.
\end{observation}

\begin{lemma}[Coefficient Identities]\label{lem:coeff-identities}
The coefficients satisfy:
\begin{enumerate}[label=(\alph*)]
\item $a/c = 1/\varphi$
\item $b/a = 1/\varphi$
\item $b + c = (2\varphi - 1)/2 = \sqrt{5}/2$
\item $c - b = 1/2 = a$
\item $bc = (\varphi - 1)\varphi/4 = (\varphi^2 - \varphi)/4 = 1/4$
\item $b^2 + a^2 = (3 - \varphi)/4$
\item $c^2 + a^2 = (\varphi + 2)/4$
\end{enumerate}
\end{lemma}

\begin{proof}
(a) $a/c = (1/2)/(\varphi/2) = 1/\varphi$. (b) $b/a = ((\varphi-1)/2)/(1/2) = \varphi - 1 = 1/\varphi$. (c) $b + c = (\varphi-1)/2 + \varphi/2 = (2\varphi - 1)/2$. Since $2\varphi - 1 = 2 \cdot (1+\sqrt{5})/2 - 1 = \sqrt{5}$, we get $b + c = \sqrt{5}/2$. (d)-(g) follow similarly. \qed
\end{proof}

\subsection{Matrix Construction}

\begin{construction}[The $\varphi$-Coupled Folding Matrix]\label{con:matrix}
The $8 \times 8$ folding matrix $U$ has the block structure:
\[
U = \begin{pmatrix} U_L \\ U_R \end{pmatrix}
\]
where $U_L$ (rows 0--3) is the \textbf{H4L block} and $U_R$ (rows 4--7) is the \textbf{H4R block}.

The explicit matrix is:
\[
U = \begin{pmatrix}
a & b & a & b & a & -b & a & -b \\
a & a & -b & -b & -a & -a & b & b \\
a & -b & -a & b & a & -b & -a & b \\
a & -a & b & -b & -a & a & -b & b \\
c & a & c & a & c & -a & c & -a \\
c & c & -a & -a & -c & -c & a & a \\
c & -a & -c & a & c & -a & -c & a \\
c & -c & a & -a & -c & c & -a & a
\end{pmatrix}
\]
\end{construction}

\begin{observation}[Entry Pattern]
Each row of $U_L$ contains exactly 4 entries of magnitude $|a| = 1/2$ and 4 entries of magnitude $|b| = (\varphi-1)/2$.

Each row of $U_R$ contains exactly 4 entries of magnitude $|c| = \varphi/2$ and 4 entries of magnitude $|a| = 1/2$.
\end{observation}

\subsection{Numerical Values}

For reference, the numerical matrix (to 6 decimal places):
\[
U \approx \begin{pmatrix}
0.5 & 0.309 & 0.5 & 0.309 & 0.5 & -0.309 & 0.5 & -0.309 \\
0.5 & 0.5 & -0.309 & -0.309 & -0.5 & -0.5 & 0.309 & 0.309 \\
0.5 & -0.309 & -0.5 & 0.309 & 0.5 & -0.309 & -0.5 & 0.309 \\
0.5 & -0.5 & 0.309 & -0.309 & -0.5 & 0.5 & -0.309 & 0.309 \\
0.809 & 0.5 & 0.809 & 0.5 & 0.809 & -0.5 & 0.809 & -0.5 \\
0.809 & 0.809 & -0.5 & -0.5 & -0.809 & -0.809 & 0.5 & 0.5 \\
0.809 & -0.5 & -0.809 & 0.5 & 0.809 & -0.5 & -0.809 & 0.5 \\
0.809 & -0.809 & 0.5 & -0.5 & -0.809 & 0.809 & -0.5 & 0.5
\end{pmatrix}
\]

% ============================================================
% SECTION 4: PROOFS OF THE MAIN THEOREM
% ============================================================
\section{Four Proofs of the $\sqrt{5}$-Coupling Theorem}

We present four independent proofs of the main theorem, each illuminating different aspects of the structure.

\subsection{Proof 1: Direct Computation}

\begin{theorem}[$\sqrt{5}$-Coupling, First Proof]\label{thm:main-proof1}
$\norm{U_L} \cdot \norm{U_R} = \sqrt{5}$
\end{theorem}

\begin{proof}
\textbf{Step 1: Compute $\norm{U_L}^2$.}

Each row of $U_L$ has 4 entries of magnitude $a = 1/2$ and 4 entries of magnitude $b = (\varphi-1)/2$:
\begin{align}
\norm{U_L}^2 &= 4a^2 + 4b^2 \\
&= 4 \cdot \frac{1}{4} + 4 \cdot \frac{(\varphi-1)^2}{4} \\
&= 1 + (\varphi - 1)^2
\end{align}

Now $(\varphi - 1)^2 = \varphi^2 - 2\varphi + 1 = (\varphi + 1) - 2\varphi + 1 = 2 - \varphi$.

Therefore:
\[
\norm{U_L}^2 = 1 + (2 - \varphi) = 3 - \varphi = \frac{5 - \sqrt{5}}{2} \approx 1.382
\]

\textbf{Step 2: Compute $\norm{U_R}^2$.}

Each row of $U_R$ has 4 entries of magnitude $c = \varphi/2$ and 4 entries of magnitude $a = 1/2$:
\begin{align}
\norm{U_R}^2 &= 4c^2 + 4a^2 \\
&= 4 \cdot \frac{\varphi^2}{4} + 4 \cdot \frac{1}{4} \\
&= \varphi^2 + 1 = (\varphi + 1) + 1 = \varphi + 2 = \frac{5 + \sqrt{5}}{2} \approx 3.618
\end{align}

\textbf{Step 3: Compute the product.}

We need to show $(3 - \varphi)(\varphi + 2) = 5$:
\begin{align}
(3 - \varphi)(\varphi + 2) &= 3\varphi + 6 - \varphi^2 - 2\varphi \\
&= \varphi + 6 - \varphi^2 \\
&= \varphi + 6 - (\varphi + 1) \qquad \text{(using $\varphi^2 = \varphi + 1$)} \\
&= 5
\end{align}

Therefore $\norm{U_L} \cdot \norm{U_R} = \sqrt{(3-\varphi)(\varphi+2)} = \sqrt{5}$. \qed
\end{proof}

\subsection{Proof 2: Difference of Squares}

\begin{theorem}[$\sqrt{5}$-Coupling, Second Proof]
$\norm{U_L} \cdot \norm{U_R} = \sqrt{5}$
\end{theorem}

\begin{proof}
Express the squared norms in terms of $\sqrt{5}$:
\begin{align}
\norm{U_L}^2 &= 3 - \varphi = 3 - \frac{1+\sqrt{5}}{2} = \frac{6 - 1 - \sqrt{5}}{2} = \frac{5 - \sqrt{5}}{2} \\
\norm{U_R}^2 &= \varphi + 2 = \frac{1+\sqrt{5}}{2} + 2 = \frac{1 + \sqrt{5} + 4}{2} = \frac{5 + \sqrt{5}}{2}
\end{align}

Now:
\begin{align}
\norm{U_L}^2 \cdot \norm{U_R}^2 &= \frac{5 - \sqrt{5}}{2} \cdot \frac{5 + \sqrt{5}}{2} \\
&= \frac{(5 - \sqrt{5})(5 + \sqrt{5})}{4} \\
&= \frac{25 - 5}{4} = \frac{20}{4} = 5
\end{align}

This is a difference of squares: $(5)^2 - (\sqrt{5})^2 = 25 - 5 = 20$. Therefore $\norm{U_L} \cdot \norm{U_R} = \sqrt{5}$. \qed
\end{proof}

\subsection{Proof 3: Via $\varphi$-Scaling}

\begin{theorem}[$\sqrt{5}$-Coupling, Third Proof]
$\norm{U_L} \cdot \norm{U_R} = \sqrt{5}$
\end{theorem}

\begin{proof}
We will show that $U_L = (1/\varphi) U_R$, which implies $\norm{U_L} = \norm{U_R}/\varphi$.

By Lemma \ref{lem:coeff-identities}(a) and (b), we have $a/c = b/a = 1/\varphi$. Since every entry in $U_L$ is either $\pm a$ or $\pm b$, and every corresponding entry in $U_R$ is either $\pm c$ or $\pm a$ (with matching signs), we have:
\[
(U_L)_{ij} = \frac{1}{\varphi} (U_R)_{ij}
\]

Therefore:
\begin{align}
\norm{U_L} \cdot \norm{U_R} &= \frac{\norm{U_R}}{\varphi} \cdot \norm{U_R} \\
&= \frac{\norm{U_R}^2}{\varphi} \\
&= \frac{\varphi + 2}{\varphi} \\
&= 1 + \frac{2}{\varphi} = 1 + 2(\varphi - 1) = 2\varphi - 1 = \sqrt{5}
\end{align}

The last step uses $2\varphi - 1 = 2 \cdot (1+\sqrt{5})/2 - 1 = \sqrt{5}$. \qed
\end{proof}

\subsection{Proof 4: Characteristic Polynomial}

\begin{theorem}[$\sqrt{5}$-Coupling, Fourth Proof]
$\norm{U_L} \cdot \norm{U_R} = \sqrt{5}$
\end{theorem}

\begin{proof}
Let $x = \norm{U_L}^2 = 3 - \varphi$ and $y = \norm{U_R}^2 = \varphi + 2$. We show $xy = 5$ using the minimal polynomial of $\varphi$.

Since $\varphi$ satisfies $\varphi^2 - \varphi - 1 = 0$, we can write:
\begin{align}
x &= 3 - \varphi \\
y &= \varphi + 2
\end{align}

Then:
\[
x + y = (3 - \varphi) + (\varphi + 2) = 5
\]

And:
\[
xy = (3 - \varphi)(\varphi + 2) = 3\varphi + 6 - \varphi^2 - 2\varphi = \varphi + 6 - (\varphi + 1) = 5
\]

So $x$ and $y$ are roots of $t^2 - 5t + 5 = 0$, which factors over $\Q(\sqrt{5})$ as $(t - x)(t - y)$ with $xy = 5$.

Therefore $\norm{U_L} \cdot \norm{U_R} = \sqrt{xy} = \sqrt{5}$. \qed
\end{proof}

% ============================================================
% SECTION 5: THE PHI-SCALING THEOREM
% ============================================================
\section{The $\varphi$-Scaling Theorem}

The $\sqrt{5}$-coupling is actually a consequence of a more fundamental structural property.

\begin{theorem}[$\varphi$-Scaling]\label{thm:phi-scaling}
The H4L and H4R blocks are related by:
\[
U_L = \frac{1}{\varphi} \cdot U_R
\]
That is, for all $i \in \{0,1,2,3\}$ and $j \in \{0,\ldots,7\}$:
\[
(U_L)_{i,j} = \frac{1}{\varphi} \cdot (U_R)_{i,j}
\]
where rows are indexed $0$--$3$ for $U_L$ and $0$--$3$ for $U_R$ (corresponding to rows $4$--$7$ of $U$).
\end{theorem}

\begin{proof}
The coefficient mapping between blocks is:
\begin{itemize}
\item Entries with $\pm a$ in $U_L$ correspond to entries with $\pm c$ in $U_R$
\item Entries with $\pm b$ in $U_L$ correspond to entries with $\pm a$ in $U_R$
\end{itemize}

The ratios are:
\[
\frac{a}{c} = \frac{1/2}{\varphi/2} = \frac{1}{\varphi}, \qquad \frac{b}{a} = \frac{(\varphi-1)/2}{1/2} = \varphi - 1 = \frac{1}{\varphi}
\]

Both ratios equal $1/\varphi$, so every entry scales by the same factor. \qed
\end{proof}

\begin{corollary}[Consequences of $\varphi$-Scaling]\label{cor:scaling-consequences}
\begin{enumerate}[label=(\alph*)]
\item $\norm{U_L} = \norm{U_R}/\varphi$
\item $\inner{U_L}{U_R} = \norm{U_R}^2/\varphi = (\varphi + 2)/\varphi = \sqrt{5}$
\item After normalizing to unit row norms, $\hat{U}_L = \hat{U}_R$ (identical rows)
\item The ``two'' H4 copies in the folding are geometrically the same, viewed at different scales
\end{enumerate}
\end{corollary}

\begin{proof}
(a) Immediate from $U_L = U_R/\varphi$. (b) $\inner{U_L}{U_R} = (U_R/\varphi) \cdot U_R = \norm{U_R}^2/\varphi$. (c) Normalization divides by norm, but $U_L$ and $U_R$ point in the same direction. (d) Follows from (c). \qed
\end{proof}

\begin{remark}[Geometric Interpretation]
The $\varphi$-scaling reveals that E8$\to$H4 folding doesn't produce two independent H4 copies; it produces \emph{one} H4 structure at two self-similar scales. This is a hallmark of golden-ratio geometry.
\end{remark}

% ============================================================
% SECTION 6: WORKED EXAMPLES
% ============================================================
\section{Worked Examples}

\subsection{Example: Computing Row 0 Norm}

\begin{example}
Compute $\norm{\text{Row}_0}$ for the H4L block.
\end{example}

\textbf{Solution.} Row 0 of $U_L$ is:
\[
\text{Row}_0 = (a, b, a, b, a, -b, a, -b) = \left(\frac{1}{2}, \frac{\varphi-1}{2}, \frac{1}{2}, \frac{\varphi-1}{2}, \frac{1}{2}, -\frac{\varphi-1}{2}, \frac{1}{2}, -\frac{\varphi-1}{2}\right)
\]

The squared norm is:
\begin{align}
\norm{\text{Row}_0}^2 &= a^2 + b^2 + a^2 + b^2 + a^2 + b^2 + a^2 + b^2 \\
&= 4a^2 + 4b^2 \\
&= 4 \cdot \frac{1}{4} + 4 \cdot \frac{(\varphi-1)^2}{4} \\
&= 1 + (\varphi - 1)^2
\end{align}

Now we compute $(\varphi - 1)^2$:
\[
(\varphi - 1)^2 = \left(\frac{1+\sqrt{5}}{2} - 1\right)^2 = \left(\frac{\sqrt{5}-1}{2}\right)^2 = \frac{5 - 2\sqrt{5} + 1}{4} = \frac{6 - 2\sqrt{5}}{4} = \frac{3 - \sqrt{5}}{2}
\]

Therefore:
\[
\norm{\text{Row}_0}^2 = 1 + \frac{3 - \sqrt{5}}{2} = \frac{2 + 3 - \sqrt{5}}{2} = \frac{5 - \sqrt{5}}{2} = 3 - \varphi \approx 1.382
\]

And $\norm{\text{Row}_0} = \sqrt{3 - \varphi} \approx 1.176$. \qed

\subsection{Example: Verifying $\varphi$-Scaling}

\begin{example}
Verify that $\text{Row}_0(U_L) = (1/\varphi) \cdot \text{Row}_0(U_R)$.
\end{example}

\textbf{Solution.}
\begin{align}
\text{Row}_0(U_L) &= (a, b, a, b, a, -b, a, -b) \\
\text{Row}_0(U_R) &= (c, a, c, a, c, -a, c, -a)
\end{align}

Check entry by entry:
\begin{itemize}
\item Position 0: $a/c = (1/2)/(\varphi/2) = 1/\varphi$ \checkmark
\item Position 1: $b/a = ((\varphi-1)/2)/(1/2) = \varphi - 1 = 1/\varphi$ \checkmark
\item Positions 2--7: Same ratios with matching signs \checkmark
\end{itemize}

Therefore $\text{Row}_0(U_L) = (1/\varphi) \cdot \text{Row}_0(U_R)$. \qed

\subsection{Example: Inner Product Computation}

\begin{example}
Compute $\text{Row}_0(U_L) \cdot \text{Row}_0(U_R)$.
\end{example}

\textbf{Solution.}
\begin{align}
\text{Row}_0(U_L) \cdot \text{Row}_0(U_R) &= ac + ba + ac + ba + ac + ba + ac + ba \\
&= 4ac + 4ab \\
&= 4a(b + c) \\
&= 4 \cdot \frac{1}{2} \cdot \frac{\sqrt{5}}{2} \qquad \text{(using $b + c = \sqrt{5}/2$)} \\
&= \sqrt{5}
\end{align}

Alternatively, using $\varphi$-scaling:
\[
\text{Row}_0(U_L) \cdot \text{Row}_0(U_R) = \frac{1}{\varphi} \norm{U_R}^2 = \frac{\varphi + 2}{\varphi} = 1 + \frac{2}{\varphi} = 2\varphi - 1 = \sqrt{5}
\]
\qed

% ============================================================
% SECTION 7: COMPUTATIONAL VERIFICATION
% ============================================================
\section{Computational Verification}

All algebraic results have been verified numerically to machine precision ($\sim 10^{-15}$).

\subsection{Verification Table}

\begin{center}
\begin{tabular}{lcc}
\toprule
Identity & Algebraic Value & Numerical Verification \\
\midrule
$\varphi$ & $(1+\sqrt{5})/2$ & $1.6180339887498949$ \\
$\varphi^2 - \varphi - 1$ & $0$ & $2.22 \times 10^{-16}$ \\
$\varphi - 1/\varphi - 1$ & $0$ & $0$ \\
$a$ & $1/2$ & $0.5$ \\
$b$ & $(\varphi-1)/2$ & $0.30901699437494742$ \\
$c$ & $\varphi/2$ & $0.80901699437494745$ \\
$a/c$ & $1/\varphi$ & $0.6180339887498949$ \\
$b/a$ & $1/\varphi$ & $0.6180339887498949$ \\
$\norm{U_L}^2$ & $3 - \varphi$ & $1.3819660112501051$ \\
$\norm{U_R}^2$ & $\varphi + 2$ & $3.6180339887498949$ \\
$\norm{U_L}$ & $\sqrt{3-\varphi}$ & $1.1755705045849463$ \\
$\norm{U_R}$ & $\sqrt{\varphi+2}$ & $1.9021130325903071$ \\
$\norm{U_L} \cdot \norm{U_R}$ & $\sqrt{5}$ & $2.2360679774997898$ \\
$\sqrt{5}$ & $\sqrt{5}$ & $2.2360679774997898$ \\
Error & $0$ & $0$ \\
\bottomrule
\end{tabular}
\end{center}

\subsection{Cross-Validation}

The identity $(3-\varphi)(\varphi+2) = 5$ was verified by:
\begin{enumerate}
\item Direct algebraic expansion (Section 4.1)
\item Difference of squares (Section 4.2)
\item Characteristic polynomial (Section 4.4)
\item Numerical computation to 16 significant figures
\end{enumerate}

All methods agree to machine precision.

% ============================================================
% SECTION 8: DISCUSSION
% ============================================================
\section{Discussion}

\subsection{The Fundamental Structure}

Our analysis reveals a hierarchy of results:

\begin{enumerate}
\item \textbf{Most fundamental}: The coefficient triple $(a, b, c)$ forms a geometric sequence with ratio $\varphi$
\item \textbf{Implies}: $U_L = (1/\varphi) U_R$ ($\varphi$-scaling)
\item \textbf{Implies}: $\norm{U_L} \cdot \norm{U_R} = \sqrt{5}$ ($\sqrt{5}$-coupling)
\end{enumerate}

The $\sqrt{5}$-coupling is thus a \emph{consequence} of the self-similar structure built into the coefficient choice.

\subsection{Relationship to Prior Work}

\textbf{Conway-Sloane \cite{conway1999}}: Established the E8-icosian connection via quaternions. Our work provides the matrix-level algebraic structure.

\textbf{Moxness \cite{moxness2014}}: Introduced the folding matrix for visualization. We characterize its intrinsic algebraic properties.

\textbf{Koca et al.\ \cite{koca2001}}: Embedded H4 in W(E8) using quaternions. Our results provide complementary matrix characterization.

\textbf{Baez \cite{baez2018}}: Connected icosahedron to E8 via two constructions. Our $\sqrt{5}$-coupling may relate to his open question about the connection between constructions.

\subsection{Connections to Broader Theory}

The $\sqrt{5}$-coupling connects to several themes:

\begin{enumerate}
\item \textbf{Self-similarity}: Golden ratio structures exhibit self-similarity at scales related by $\varphi$. Our result shows this extends to folding matrices.

\item \textbf{Chirality}: The L/R decomposition in E8$\to$H4 mirrors chirality in physics. The $\sqrt{5}$ coupling constrains this relationship.

\item \textbf{Non-crystallographic geometry}: H4 is the largest non-crystallographic Coxeter group. The irrational coupling $\sqrt{5}$ reflects this.
\end{enumerate}

\subsection{Open Questions}

\begin{enumerate}
\item \textbf{Generalization}: Does every E8$\to$H4 folding matrix exhibit a similar coupling identity, or is the $\sqrt{5}$-coupling specific to the $\varphi$-coupled family?

\item \textbf{Other foldings}: What are the analogous coupling identities for E6$\to$G2, E7$\to$?, and other exceptional foldings?

\item \textbf{Physical interpretation}: Does the $\sqrt{5}$-coupling have physical significance in E8-based theories?

\item \textbf{Baez connection}: How does the $\sqrt{5}$-coupling relate to Baez's two constructions of E8 from the icosahedron?
\end{enumerate}

\subsection{Conclusion}

We have proven that the $\varphi$-coupled E8$\to$H4 folding matrix exhibits a precise $\sqrt{5}$-coupling between its chiral components:
\[
\norm{U_L} \cdot \norm{U_R} = \sqrt{5}
\]

More fundamentally, the blocks satisfy $U_L = (1/\varphi) U_R$, revealing that the ``two'' H4 copies are geometrically identical at different scales---a manifestation of golden ratio self-similarity.

To our knowledge, the specific identities $\norm{U_L} \cdot \norm{U_R} = \sqrt{5}$ and $U_L = (1/\varphi) U_R$ have not been previously published.

% ============================================================
% ACKNOWLEDGMENTS
% ============================================================
\section*{Acknowledgments}

The author thanks J.G. Moxness for foundational work on E8$\to$H4 folding matrices, and the mathematical physics community for the rich literature on E8 and icosahedral symmetry.

% ============================================================
% APPENDICES
% ============================================================
\appendix

\section{Complete Derivation Details}\label{app:derivations}

\subsection{Derivation of $(\varphi - 1)^2 = 2 - \varphi$}

Starting from $\varphi^2 = \varphi + 1$:
\begin{align}
(\varphi - 1)^2 &= \varphi^2 - 2\varphi + 1 \\
&= (\varphi + 1) - 2\varphi + 1 \\
&= 2 - \varphi
\end{align}

Numerically: $(\varphi - 1)^2 = (0.618...)^2 = 0.382... = 2 - 1.618... = 2 - \varphi$. \checkmark

\subsection{Derivation of $b + c = \sqrt{5}/2$}

\begin{align}
b + c &= \frac{\varphi - 1}{2} + \frac{\varphi}{2} \\
&= \frac{2\varphi - 1}{2}
\end{align}

Now $2\varphi - 1 = 2 \cdot \frac{1+\sqrt{5}}{2} - 1 = 1 + \sqrt{5} - 1 = \sqrt{5}$.

Therefore $b + c = \sqrt{5}/2$. \checkmark

\subsection{Derivation of $(3-\varphi)(\varphi+2) = 5$}

Method 1 (expansion):
\begin{align}
(3-\varphi)(\varphi+2) &= 3\varphi + 6 - \varphi^2 - 2\varphi \\
&= \varphi + 6 - \varphi^2 \\
&= \varphi + 6 - (\varphi + 1) \\
&= 5
\end{align}

Method 2 (numerical): $(3 - 1.618...)(1.618... + 2) = (1.382...)(3.618...) = 5.000...$. \checkmark

\section{Python Verification Code}\label{app:code}

\begin{lstlisting}
import numpy as np

# Golden ratio
phi = (1 + np.sqrt(5)) / 2

# Coefficients
a = 0.5
b = (phi - 1) / 2
c = phi / 2

# Verify coefficient relationships
print(f"a/c = {a/c:.15f}")
print(f"1/phi = {1/phi:.15f}")
print(f"b/a = {b/a:.15f}")
print(f"Difference: {abs(a/c - 1/phi):.2e}")

# Construct the matrix
U = np.array([
    [a, b, a, b, a, -b, a, -b],
    [a, a, -b, -b, -a, -a, b, b],
    [a, -b, -a, b, a, -b, -a, b],
    [a, -a, b, -b, -a, a, -b, b],
    [c, a, c, a, c, -a, c, -a],
    [c, c, -a, -a, -c, -c, a, a],
    [c, -a, -c, a, c, -a, -c, a],
    [c, -c, a, -a, -c, c, -a, a]
])

# Split into blocks
U_L = U[:4, :]
U_R = U[4:, :]

# Compute row norms
norm_L = np.linalg.norm(U_L[0, :])
norm_R = np.linalg.norm(U_R[0, :])

print(f"\n||U_L|| = {norm_L:.15f}")
print(f"sqrt(3-phi) = {np.sqrt(3-phi):.15f}")
print(f"||U_R|| = {norm_R:.15f}")
print(f"sqrt(phi+2) = {np.sqrt(phi+2):.15f}")

# Verify main theorem
product = norm_L * norm_R
print(f"\n||U_L|| * ||U_R|| = {product:.15f}")
print(f"sqrt(5) = {np.sqrt(5):.15f}")
print(f"Difference: {abs(product - np.sqrt(5)):.2e}")

# Verify phi-scaling
scaling_check = U_L / U_R
print(f"\nU_L / U_R (should be 1/phi):")
print(f"Mean: {np.mean(np.abs(scaling_check)):.15f}")
print(f"1/phi = {1/phi:.15f}")

# Verify (3-phi)(phi+2) = 5
algebraic = (3 - phi) * (phi + 2)
print(f"\n(3-phi)(phi+2) = {algebraic:.15f}")
print(f"Difference from 5: {abs(algebraic - 5):.2e}")
\end{lstlisting}

\textbf{Output:}
\begin{verbatim}
a/c = 0.618033988749895
1/phi = 0.618033988749895
b/a = 0.618033988749895
Difference: 0.00e+00

||U_L|| = 1.175570504584946
sqrt(3-phi) = 1.175570504584946
||U_R|| = 1.902113032590307
sqrt(phi+2) = 1.902113032590307

||U_L|| * ||U_R|| = 2.236067977499790
sqrt(5) = 2.236067977499790
Difference: 0.00e+00

U_L / U_R (should be 1/phi):
Mean: 0.618033988749895
1/phi = 0.618033988749895

(3-phi)(phi+2) = 5.000000000000000
Difference from 5: 0.00e+00
\end{verbatim}

\section{Tables of Computed Values}\label{app:tables}

\subsection{Golden Ratio Powers}

\begin{center}
\begin{tabular}{cccc}
\toprule
$n$ & $\varphi^n$ & Decimal & Fibonacci Form \\
\midrule
$-3$ & $(2-\varphi)/\varphi$ & $0.236$ & $2F_{-3} + F_{-4}$ \\
$-2$ & $2 - \varphi$ & $0.382$ & $2 - \varphi$ \\
$-1$ & $\varphi - 1$ & $0.618$ & $\varphi - 1$ \\
$0$ & $1$ & $1.000$ & $1$ \\
$1$ & $\varphi$ & $1.618$ & $\varphi$ \\
$2$ & $\varphi + 1$ & $2.618$ & $\varphi + 1$ \\
$3$ & $2\varphi + 1$ & $4.236$ & $2\varphi + 1$ \\
$4$ & $3\varphi + 2$ & $6.854$ & $3\varphi + 2$ \\
$5$ & $5\varphi + 3$ & $11.090$ & $5\varphi + 3$ \\
\bottomrule
\end{tabular}
\end{center}

\subsection{Key Algebraic Values}

\begin{center}
\begin{tabular}{cccl}
\toprule
Symbol & Algebraic & Decimal & Description \\
\midrule
$a$ & $1/2$ & $0.500$ & Coefficient \\
$b$ & $(\varphi-1)/2$ & $0.309$ & Coefficient \\
$c$ & $\varphi/2$ & $0.809$ & Coefficient \\
$\norm{U_L}^2$ & $3-\varphi$ & $1.382$ & H4L row norm squared \\
$\norm{U_R}^2$ & $\varphi+2$ & $3.618$ & H4R row norm squared \\
$\norm{U_L}$ & $\sqrt{3-\varphi}$ & $1.176$ & H4L row norm \\
$\norm{U_R}$ & $\sqrt{\varphi+2}$ & $1.902$ & H4R row norm \\
$\norm{U_L}\norm{U_R}$ & $\sqrt{5}$ & $2.236$ & Coupling constant \\
\bottomrule
\end{tabular}
\end{center}

\subsection{The $\varphi$-Hierarchy}

\begin{center}
\begin{tabular}{cccl}
\toprule
Value & Decimal & Algebraic Form & Role in E8$\to$H4 \\
\midrule
$\varphi^{-2}$ & $0.382$ & $2-\varphi$ & Scaling factor \\
$\varphi^{-1}$ & $0.618$ & $\varphi-1$ & Coefficient ratio \\
$1$ & $1.000$ & $1$ & Unit scale \\
$\sqrt{3-\varphi}$ & $1.176$ & $\sqrt{(5-\sqrt{5})/2}$ & \textbf{H4L row norm} \\
$\sqrt{2}$ & $1.414$ & $\sqrt{2}$ & E8 root norm \\
$\varphi$ & $1.618$ & $(1+\sqrt{5})/2$ & Golden ratio \\
$\sqrt{\varphi+2}$ & $1.902$ & $\sqrt{(5+\sqrt{5})/2}$ & \textbf{H4R row norm} \\
$\sqrt{5}$ & $2.236$ & $\sqrt{5}$ & \textbf{Coupling constant} \\
$\varphi^2$ & $2.618$ & $\varphi+1$ & Golden ratio squared \\
\bottomrule
\end{tabular}
\end{center}

% ============================================================
% BIBLIOGRAPHY
% ============================================================
\begin{thebibliography}{99}

\bibitem{conway1999}
J.H. Conway and N.J.A. Sloane,
\textit{Sphere Packings, Lattices and Groups},
Grundlehren der mathematischen Wissenschaften, vol.~290, 3rd ed.
Springer, New York, 1999.
ISBN: 978-0-387-98585-5.
DOI: \href{https://doi.org/10.1007/978-1-4757-6568-7}{10.1007/978-1-4757-6568-7}.

\bibitem{coxeter1973}
H.S.M. Coxeter,
\textit{Regular Polytopes}, 3rd ed.
Dover Publications, New York, 1973.
ISBN: 978-0-486-61480-9.

\bibitem{viazovska2017}
M. Viazovska,
``The sphere packing problem in dimension 8,''
\textit{Annals of Mathematics}, vol.~185, no.~3, pp.~991--1015, 2017.
DOI: \href{https://doi.org/10.4007/annals.2017.185.3.7}{10.4007/annals.2017.185.3.7}.

\bibitem{moxness2014}
J.G. Moxness,
``The 3D Visualization of E8 using an H4 Folding Matrix,''
viXra:1411.0130, 2014.
Available: \url{https://vixra.org/abs/1411.0130}.

\bibitem{koca2001}
M. Koca, R. Ko\c{c}, and M. Al-Barwani,
``Noncrystallographic Coxeter group H4 in E8,''
\textit{Journal of Physics A: Mathematical and General}, vol.~34, no.~50, pp.~11201--11213, 2001.
DOI: \href{https://doi.org/10.1088/0305-4470/34/50/303}{10.1088/0305-4470/34/50/303}.

\bibitem{denney2019}
T. Denney, D. Hooker, D. Johnson, T. Robinson, M. Butler, and S. Claiborne,
``The Geometry of H4 Polytopes,''
arXiv:1912.06156 [math.MG], 2019.
DOI: \href{https://doi.org/10.48550/arXiv.1912.06156}{10.48550/arXiv.1912.06156}.

\bibitem{baez2018}
J.C. Baez,
``From the Icosahedron to E8,''
\textit{London Mathematical Society Newsletter}, no.~476, pp.~18--23, 2018.
arXiv: \href{https://arxiv.org/abs/1712.06436}{1712.06436}.

\bibitem{humphreys1990}
J.E. Humphreys,
\textit{Reflection Groups and Coxeter Groups},
Cambridge Studies in Advanced Mathematics, vol.~29.
Cambridge University Press, 1990.
ISBN: 978-0-521-43613-7.

\bibitem{moody1988}
R.V. Moody and J. Patera,
``Quasicrystals and icosians,''
\textit{Journal of Physics A: Mathematical and General}, vol.~26, pp.~2829--2853, 1993.
DOI: \href{https://doi.org/10.1088/0305-4470/26/12/022}{10.1088/0305-4470/26/12/022}.

\bibitem{sadoc2001}
J.F. Sadoc and R. Mosseri,
\textit{Geometrical Frustration}.
Cambridge University Press, 2001.
ISBN: 978-0-521-44198-8.

\end{thebibliography}

\end{document}
